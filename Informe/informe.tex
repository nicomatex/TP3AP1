\documentclass[12pt]{article}
% pre\'ambulo

\usepackage{lmodern}
\usepackage[T1]{fontenc}
\usepackage[spanish,activeacute]{babel}
\usepackage[utf8]{inputenc}
\usepackage{listings}
\usepackage{graphicx}
\graphicspath{ {imagenes/} }

\usepackage{mathtools}
\usepackage{sectsty}


\sectionfont{\clearpage}

\title{Trabajo pr\'actico Nº 3. Simulador de batallas de Gunplas.}
\author{Nicol\'as Federico Aguerre\\Padr\'on 102145
		\and
		Juan Pablo Di Como\\Padr\'on 102155}


\begin{document}
% cuerpo del documento

	\maketitle

	Algoritmos y Programaci\'on I

	C\'atedra: Essaya 


	Pr\'actica: Alan

	Corrector: Ayel\'en Bibiloni Lombardi

	\section{Introducci\'on}
		El objetivo del presente trabajo pr\'actico es implementar un simulador de combate del estilo de New Gundam Breaker, as\'i como 
		tambi\'en una inteligencia artificial capaz de participar en la simulaci\'on, actuando de "jugador", ambos implementados en Python.

	\section{Dise'no del simulador}
		En primer lugar, teniendo en cuenta la interfaz minima provista en las consignas del trabajo, se dise'nan las clases basicas de tal forma que cumplan la interfaz pedida.
		La estructura decidida para manejar de forma adecuada los pilotos y gunplas participantes en cada combate se basa en una divisi'on por equipos, siguiendo la siguiente l'ogica:

		\begin{itemize}
			\item Cada piloto tiene asociado un 'unico gunpla. 
			\item A su vez, cada piloto es representado por un objeto $Participante$ dentro del simulador.
			\item Los participantes estan agrupados dentro de $Equipos$, teniendo cada equipo exactamente la misma 
			cantidad $n$ de participantes.
		\end{itemize}

		La organizaci'on mencionada se implementa utilizando clases y objetos, debido a que se decidi'o que esta era la estructura 
		que aportaba mayor legibilidad y facilidad de mantenimiento al c'odigo del programa.\\
		\par
		Otro punto importante en el dise'no del programa es que todos los valores asociados al funcionamiento(Es decir, la cantidad de pilotos, cantidad de equipos, y rangos entre los cuales pueden variar los $stats$ de los gunpla) pudieran ser modificados
		de forma simple, sin necesidad de alterar el c'odigo, lo cual aporta una mayor din'amica al c'odigo. Los par'ametros mencionados 
		se ubican en la cabecera de cada uno de los archivos que componen al proyecto en forma de constantes.\\
		\par

		El ciclo de funcionamiento del programa se estructura de la siguiente forma:

		\begin{enumerate}
			\item Se generan los equipos con sus respectivos participantes. 
			\item Se generan las partes que se van a utilizar para configurar a los gunplas (esqueletos, partes, armas).
			\item Se inicia el ciclo de seleccion de partes por parte de los pilotos, dividido en 2 etapas:
			\begin{enumerate}
				\item Se les ofrecen a los pilotos ciertas partes disponibles para reservar, de a una por vez, hasta que ya no hay mas partes que reservar.
				\item Los pilotos eligen, de entre todas las partes reservadas, aquellas que desean conservar.
			\end{enumerate}
			\item Se equipan los gunplas asociados a cada piloto con las partes reservadas por estos.
			\item Se ordenan (Segun lo especificado por las consignas) los participantes en una cola de turnos seg'un la velocidad
			de sus gunplas.
			\item Se inicializa el ciclo de juego(Ver detalle del ciclo de juego mas adelante.)
		\end{enumerate}

		Para organizar las partes en el paso $2$, se utiliza un diccionario de la forma \{tipo de parte: pila con partes\} que tiene por claves 
		cadenas que representan tipos de parte, y por valores, pilas de partes de este mismo tipo. Esto permite que cada piloto, cuando es su turno de reservar partes, pueda elegir exactamente de a una parte de cada tipo (las que se encuentran en el tope de cada pila). Para organizar los esqueletos se utiliza simplemente una lista, debido a que m'ultiples pilotos pueden elegir el mismo esqueleto. 
		Adem'as, para guardar las partes reservadas por cada piloto, se utiliza un diccionario de la forma \{piloto:partes reservadas\}
		que tiene por claves a cada piloto, y por valores a la lista de partes reservadas por ese piloto. A pesar de que los pilotos son objetos, y por tanto mutables, es posible utilizarlos como claves en el diccionario debido a que , por defecto, python guarda como claves las direcciones de memoria de los objetos, convirtiendo estas claves en inmutables.o
\end{document}